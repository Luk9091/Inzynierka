\section{Problemy konstrukcyjne}
\label{sec:problemy_konstrukcyjne}
    W poniższym rozdziale zostaną przedstawione problemy konstrukcyjne,
    które pojawiły się podczas realizacji projektu.

    \subsection{Jazda prosto}
    \label{section:jazda_prosto}
        Do jednej z najprostszych czynności pojazdów należy jazda prosto.
        Operacja ta może wydawać się łatwa, jednak wcale taka nie jest.
        Przykładowo, kiedy usiądziemy za kierownicą prawdziwego samochodu, kontrolujemy podświadomie wiele parametrów, takich jak:
        \begin{itemize}
            \item nachylenie terenu,
            \item aktualną pozycję kół,
            \item linie poziome na drodze czy pobocze drogi.
        \end{itemize}
        W przeciwieństwie do ludzi, którzy często oceniają większość parametrów ,,na wyczucie'', pojazdy autonomiczne,
        będące de facto robotami, nie mogą opierać się na odczuciu prawidłowości toru jazdy.
        Z tego względu niezbędne jest zastosowanie odpowiednich czujników, umożliwiających precyzyjną kontrolę ruchu.
        Poniżej pokrótce zostaną opisane rozwiązania, dzięki którym samochód jest w stanie jechać w linii prostej.

        \subsubsection{Różnica prędkości silników}
            Zastosowanie dwóch silników ma swoje wady i zalety.
            Olbrzymim atutem jest podwojenie mocy pojazdu.
            Jednak wielkim minusem okazało się sterowanie i nierównomierność pracy.
            Na wykresie \ref{plot:distance_err_in_time_const_speed} zaprezentowano różnicę drogi pokonanej przez każdy z nich dla jednakowego sygnału sterującego.

            \begin{figure}[!ht]
                \centering
                \begin{tikzpicture}
                    \begin{axis}[
                        width = 0.7\textwidth,
                        grid = both,
                        grid style = dashed,
                        % axis lines = middle,
                        xlabel = czas ${[s]}$,
                        ylabel = różnica odległości ${[mm]}$,
                        xmin = 0,
                        xmax = 20,
                    ]
                        \addplot[blue] table[x = Time, y = Diff, col sep = comma]{Measure/distance_no_PID_speed_50.csv};
                        \legend{Błąd odległości}
                    \end{axis}
                \end{tikzpicture}
                \caption{Wykres różnicy przebytej drogi między prawym a lewym silnikiem bez regulacji}
                \label{plot:distance_err_in_time_const_speed}
            \end{figure}

            Aby rozwiązać powyższy problem, wykorzystano regulator PID, pozwalający w trakcie pracy dostosowywać szybkość silników.
            W celu zminimalizowania początkowego błędu, podczas startu została zastosowana sztywna korekta prędkości.
            % Natomiast podczas startu została zastosowana sztywna korekta prędkości, w celu zminimalizowania początkowego błędu.
            \begin{figure}[!ht]
                \centering
                \begin{tikzpicture}
                    \begin{axis}[
                        width = 0.7\textwidth,
                        grid = both,
                        grid style = dashed,
                        xlabel = czas ${[s]}$,
                        ylabel = różnica odległości ${[mm]}$,
                        xmin = 0,
                        xmax = 60,
                    ]
                        \addplot[blue] table[x = Time, y = Distance_err, col sep = comma]{Measure/PID_speed_10.csv};
                        \legend{Błąd odległości}
                    \end{axis}
                \end{tikzpicture}
                \caption{Wykres różnicy przebytej drogi między prawym a lewym silnikiem z włączonym regulatorem PID. Prędkość pojazdu: $(35.10 \pm 0.10)\frac{mm}{s}$.}
                \label{plot:PID_distance_err_in_time}
            \end{figure}

            Jak widać na wykresie (rys. \ref{plot:PID_distance_err_in_time}), we wstępnej fazie silniki nadal nie pracują równomiernie, jednak w dłuższej perspektywie różnica odległości oscyluje wokół  $ 0\pm 3mm$.

    \subsubsection{Nieidealność podwozia}
        Kolejnym napotkanym problemem, okazała się nieidealność podwozia,
        szczególnie elementu przedstawionego na rysunku \ref{fig:frontAxis_model} oraz nierównomierne rozłożenie masy względem środka.
        Wszystkie wyżej wymienione niedokładności powodowały, że pojazd cały czas skręcał w~jedną stronę.
        Rozwiązaniem było zastosowanie pętli sprzężenia zwrotnego, opartej na czujniku $MPU6050$, który został użyty jako żyroskop, pozwalający na pomiar zmiany kąta.
        Dzięki niemu można korygować obrót samochodu.

        % Jednak jego wykorzystanie nie było bezproblemowe i wymagało dodatkowych poprawek.
        Układ akcelerometru i żyroskopu ($MPU6050$) wymagał wstępnej kalibracji.
        Na wykresie (rys. \ref{plot:gyro_magneto_measure}) zaprezentowano pomiar kąta dla obrotu pojazdu ze stałą prędkością w jednym kierunku, po wprowadzeniu poprawki do odczytanych danych.
        Poniższe zestawienie zostało wykonane w celu ustalenia liniowości pracy tego modułu.
        Źródło referencyjne stanowił uprzednio skalibrowany magnetometr ($QMC5883L$).
        Na oba czujniki został nałożony filtr dolnoprzepustowy, eliminujący szumy.

        \begin{figure}[!ht]
            \centering
                \begin{tikzpicture}
                    \begin{axis}[
                        width = 0.65\textwidth,
                        grid = both,
                        grid style = dashed,
                        xlabel = czas ${[s]}$,
                        ylabel = kąt zmierzony podczas obrotu ${[^\circ]}$,
                        ymin = -5,
                        ymax = 365,
                        xmin = 0,
                        xmax = 2.9,
                        ytick = {0, 30, ..., 360},
                        legend style={at={(0.2, 0.92)}, anchor=north},
                    ]
                        \addplot[orange] table[x = Time, y = Compass_azimuth, col sep = comma]{Measure/angles.csv};
                        \addplot[blue] table[x = Time, y = Gyro_z, col sep = comma]{Measure/angles.csv};
                        \legend{magnetometr,żyroskop}
                    \end{axis}
                \end{tikzpicture}
                \caption{Wykres pomiaru azymutu za pomocą magnetometru oraz zmiany kąta wykazanego przez żyroskop}
                \label{plot:gyro_magneto_measure}
        \end{figure}

        Wykres \ref{plot:delta_angle_with_gyro} przedstawia narastanie kąta w funkcji czasu.
        Widać liniowość pracy żyroskopu, dzięki czemu, z dużą pewnością można określić, że pojazd porusza się w linii prostej.
        %  co pozwala zaufać temu czujnikowi na tyle, aby na jego podstawie określać czy pojazd porusza się w prostej linii.

        % \begin{wrapfig}[!ht]
        \begin{figure}[!ht]
            \centering
                \begin{tikzpicture}
                    \begin{axis}[
                        width = 0.7\textwidth,
                        grid = both,
                        grid style = dashed,
                        xlabel = czas ${[s]}$,
                        ylabel = kąt zmierzony podczas obrotu ${[^\circ]}$,
                        xmin = 0,
                        xmax = 2.9,
                    ]
                        \addplot[blue] table[x = Time, y = Delta_angle, col sep = comma]{Measure/angles.csv};
                    \end{axis}
                \end{tikzpicture}
                \caption{Wykres narastania kąta, zmierzonego przez żyroskop}
                \label{plot:delta_angle_with_gyro}
        \end{figure}
        % \end{wrapfig}


\newpage
    Oba opisane rozwiązania pozwalają na równą jazdę z niewielkim odchyleniem ($\pm 2^\circ$).
    W~sytuacjach kiedy samochód zaczyna znacząco skręcać, na przykład po zderzeniu, sprzężenie zwrotne z żyroskopu wykrywa zmianę kierunku.
    Kolejny regulator PID odczytuje jej wartość i koryguje kąt serwomechanizmu, tak aby pojazd wrócił na poprzednią trasę.

    \subsection{Skręcanie}
        Naprawienie problemów z przemieszczaniem się po prostej to tylko połowa sukcesu.
        Następne mankamenty zauważono podczas próby skręcania.
        W pierwotnej wersji ustawienie stałego kąta kół oraz przejechanie tej samej drogi nie zawsze dawało ten sam rezultat.
        Dodatkowo pojazd cały czas miał tendencję do preferowania jednego kierunku.

        \subsubsection{Dyferencjał}
            \label{subsubsec:dyferencjal}
            Najprostszy do zauważenia był brak dyferencjału.
            Tylna oś samochodu posiada dwa silniki. W poprzednich testach zostały one ,,software'owo połączone sztywną belką", co powodowało poślizg w trakcie skręcania, a w konsekwencji uniemożliwiało powtarzalność manewru.

            Wyjściem z tej sytuacji jest zmiana prędkości pracy silników w zależności od promienia skrętu.
            Poniżej przedstawiono schemat algorytmu, na którego podstawie wyznaczono procentową wartość dyferencjału.
            Rysunek \ref{draw:turning_car} ilustruje zachowanie kół podczas tego ruchu.
            \begin{figure}[!ht]
    \centering
    \begin{tikzpicture}
        \draw
            (0, 0) node[draw, rectangle, rounded corners, fill = black, minimum width = 2cm, minimum height = 0.8cm, rotate = 30](frontL){}
            (0, -3) node[draw, rectangle, rounded corners, fill = black, minimum width = 2cm, minimum height = 0.8cm, rotate = 30](frontR){}

            (6, 0) node[draw, rectangle, rounded corners, fill = black, minimum width = 2cm, minimum height = 0.8cm](backL){}
            (6, -3) node[draw, rectangle, rounded corners, fill = black, minimum width = 2cm, minimum height = 0.8cm](backR){}

            (3, 0) node[above]{Zewnętrzna strona}
            (3,-3) node[below]{Wewnętrzna strona}
        ;

        \draw[Stealth-Stealth]
            (frontL) ++ (0, 1) --node[above]{$l$}++ (6, 0)
        ;
        \draw[dashed, gray]
            (frontL) --++(0, 1)
            (backL) --++(0, 1)
        ;

        \draw[color = red]
            (backL) -- ++ (0, -10) -- ++ (0, -1) coordinate(meet) --++ (0, -1)
            (frontL) -- (meet)
            (frontR) -- (meet)
        ;
        \draw
            (meet) ++ (0, 3) arc(90:127:3) node[above]{$\alpha$}
            (meet) ++ (0, 2) arc(90:147:1) node[above]{$\beta$}
        ;

        \draw[dotted]
            (backL) to[short, l=$\Delta r$] (backR)
            (backR) ++ (0, -2.5) node[right]{$r$}
        ;

        \draw[dashed]
            (frontL) ++ (0, 1) -- (0, -5)
            (0, -4) arc(90:180:-1)
            (0.5, -4.5) node[]{$\gamma$}
            (1.2, -4.25) node[rotate = -50]{$90 - \gamma$}
        ;
    \end{tikzpicture}
    \caption{Rysunek poglądowy do skręcenia}
    \label{draw:turning_car}
\end{figure}

\newpage
            Jak widać skręt pojazdu można opisać jako ruch dwóch kół po okręgach o wspólnym środku, których poszczególne prędkości liniowe nie są sobie równe.
            Jednak występuje zależność:
            \begin{gather}
                \omega_{\text{wewnęrznego koła}} = \omega_{\text{zewnętrznego koła}}\\
                \frac{v_{\text{wewnęrznego koła}}}{r} = \frac{v_{\text{zewnętrznego koła}}}{r + \Delta r}\\
                \frac{v_\text{wewnęrznego koła}}{v_{\text{zewnętrznego koła}}} = \frac{r}{r + \Delta r}
            \end{gather}

            Aby policzyć stosunek prędkości silników, należy wyznaczyć promień skrętu.
            Do tego celu, można wykorzystać znane wartości i równanie \ref{eq:turning_radius}:
            \begin{gather}
                \tan \left(90 - \gamma\right) = \frac{l}{r}\\
                r(\gamma) = \frac{l}{\tan(90-\gamma)}
                \label{eq:turning_radius}
            \end{gather}
%
            gdzie:
            \begin{itemize}
                \item $l = 155$ -- długość między osiami samochodu,
                \item $\Delta r = 125$ -- odległość między kołami w jednej osi,
                \item $\gamma \in <60^\circ \div 120^\circ>$ -- kąt skrętu kół, ustawiany przez algorytm nawigacji lub użytkownika.
            \end{itemize}

        A zatem procentową różnicę szybkości w funkcji kąta można wyrazić jako:
        \begin{gather}
            \frac{v_{\text{prawe koła}}}{v_{\text{lewe koła}}} = 1 + \frac{\Delta r}{r} = 1 + \Delta r \cdot \frac{\tan(90 - \gamma)}{l}
            \label{eq:differential}
        \end{gather}

        Po zastosowaniu przedstawionego równania \eqref{eq:differential} w automatyce regulującej moc silników, udało się uzyskać powtarzalne wyniki podczas wykonywania tego samego manewru.

        \subsubsection{Nierównomierność skrętu}
        \label{subsec:nierownomiernosc_skretu}
            W trakcie testów okazało się, że pojazd nie skręca równomiernie w obie strony.
%
            Podczas ruchu w prawo, na łuku o długości $s = 500mm$, samochód zakreślał kąt około $70^\circ$.
            Natomiast na tym samym odcinku, ale w lewo wyznaczany kąt to około $50^\circ$.

            Wymusiło to ustawienie offsetu, przesuwającego środek osi.
            Jego wartość została otrzymana eksperymentalnie i była równa $-6^\circ$.
            Po wprowadzeniu poprawki uzyskany łuk podczas skręcania w obu kierunkach wynosił około $60^\circ$.



    Dzięki wszystkim wyżej wymienionym zabiegom udało się uzyskać powtarzalne wyniki, zarówno w czasie jazdy prosto, cofania się oraz skręcania.
    Podane działania są niezbędne, jeśli budowany robot ma być sterowany autonomicznie.

