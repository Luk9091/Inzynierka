\section{Problemy konstrukcyjne}
\label{sec:problemy_konstrukcyjne}
    W poniższym rozdziale zostaną przedstawione problemy konstrukcyjne,
    które pojawiły się podczas realizacji projektu.

    \subsection{Jazda prosto}
    \label{section:jazda_prosto}
        Najprostszą czynnością każdego pojazdu, jest jazda prosto.
        Operacja ta może wydawać się prosta, jednak wcale taka nie jest.
        Przykładowo, kiedy usiądziemy za kierownicą, prawdziwego samochodu, kontroluje podświadomie wiele parametrów takich jak:
        \begin{itemize}
            \item nachylenie terenu,
            \item aktualną pozycję kół,
            \item linie poziome na drodze czy pobocze drogi.
        \end{itemize}
        Jako ludzie, wiele tych parametrów kontroluje ,,na wyczucie''`' jednak pojazdy autonomiczne, są poniekąd robotami, dla których istnieje tylko informacja \textit{tak} lub \textit{nie}.
        Poniższy projekt podczas testów, wykazywał wiele mankamentów, które uniemożliwiały jazdę prostą.
        W poniższym rozdziale pokrótce zostaną one opisane wraz z proponowanymi metodami radzenia sobie z nie idealnością konstrukcji.
        % Były to między innymi:
        
        \subsubsection{Różnica prędkości silników}
            Zastosowanie dwóch silników, ma swoje wady i zalety.
            Niewątpliwą zaletą jest podwojenie mocy pojazdu.
            Jednak olbrzymią wadą okazało się sterowanie i nierównomierność pracy silników.
            Na wykresie \ref{plot:distance_err_in_time_const_speed} poniżej przedstawiono różnicę drogi pokonanej przez każdy silnik dla jednakowego sygnału sterującego.

            \begin{figure}[!ht]
                \centering
                \begin{tikzpicture}
                    \begin{axis}[
                        width = 0.7\textwidth,
                        grid = both,
                        grid style = dashed,
                        % axis lines = middle,
                        xlabel = czas ${[s]}$,
                        ylabel = różnica odległości ${[mm]}$,
                        xmin = 0,
                        xmax = 20,
                    ]
                        \addplot[blue] table[x = Time, y = Diff, col sep = comma]{Measure/distance_no_PID_speed_50.csv};
                        \legend{Błąd odległości}
                    \end{axis}
                \end{tikzpicture}
                \renewcommand{\figurename}{Wykres}
                \caption{Różnica przebytej drogi między prawym a lewym silnikiem bez regulacji}
                \label{plot:distance_err_in_time_const_speed}
            \end{figure}

            Aby rozwiązać powyższy problem należy zastosować regulator PID, pozwalający w trakcie pracy, korygować prędkość silników.
            Natomiast podczas startu silników została zastosowana sztywna korekta prędkości, w celu zminimalizowania błędu.
            \begin{figure}[!ht]
                \centering
                \begin{tikzpicture}
                    \begin{axis}[
                        width = 0.7\textwidth,
                        grid = both,
                        grid style = dashed,
                        xlabel = czas ${[s]}$,
                        ylabel = różnica odległości ${[mm]}$,
                        xmin = 0,
                        xmax = 60,
                    ]
                        \addplot[blue] table[x = Time, y = Distance_err, col sep = comma]{Measure/PID_speed_10.csv};
                        \legend{Błąd odległości}
                    \end{axis}
                \end{tikzpicture}
                \renewcommand{\figurename}{Wykres}
                \caption{Różnica przebytej drogi między prawym a lewym silnikiem z włączonym regulatorem PID. Prędkość pojazdu: $(35.10 \pm 0.10)\frac{mm}{s}$.}
                \label{plot:PID_distance_err_in_time}
            \end{figure}

            Jak widać na wykresie \ref{plot:PID_distance_err_in_time} w początkowej fazie, silniki nadal nie pracują równomiernie, w dłuższej perspektywie praca silników osypuje w okół stałej wartości $(0 \pm 3)mm$.

    \subsubsection{Nieidealność podwozia}
        Kolejnym napotkanym problemem, okazuje się nie idealność podwozia.
        Szczególnie elementu przedstawionego na rysunku \ref{fig:frontAxis_model} oraz nie równomierne rozłożenie masy względem środka.
        Wszystkie wyżej wymienione niedokładności powodowały, że pojazd cały czas skręcał w jedną stronę.
        Rozwiązaniem tego problemu jest zastosowanie pętli sprzężenia zwrotnego, opartej na czujniku $MPU6050$, który został użyty jak żyroskop, pozwalający na pomiar biedzącego kąta.
        Dzięki czemu można korygować kierunek skręcania pojazdu.
        
        Jednak jego wykorzystanie nie było bezproblemowe i wymagało dodatkowych poprawek.
        Na wykresie \ref{plot:gyro_magneto_measure} przedstawiono pomiar kąta dla obrotu pojazdu ze stałą prędkością w jednym kierunku po kalibracji czujnika.
        Poniższy pomiar został wykonany w celu ustalenia liniowości pracy żyroskopu opartego na akcelerometrze. 
        Układem referencyjnym był uprzednio skalibrowany magnetometr (układ $QMC5883L$).
% 
        \begin{figure}[!ht]
            \centering
                \begin{tikzpicture}
                    \begin{axis}[
                        width = 0.65\textwidth,
                        grid = both,
                        grid style = dashed,
                        xlabel = czas ${[s]}$,
                        ylabel = kąt zmierzony podczas obrotu ${[^\circ]}$,
                        ymin = -5,
                        ymax = 365,
                        xmin = 0,
                        xmax = 2.9,
                        ytick = {0, 30, ..., 360},
                        legend style={at={(0.2, 0.92)}, anchor=north},
                    ]
                        \addplot[orange] table[x = Time, y = Compass_azimuth, col sep = comma]{Measure/angles.csv};
                        \addplot[blue] table[x = Time, y = Gyro_z, col sep = comma]{Measure/angles.csv};
                        \legend{magnetometr,żyroskop}
                    \end{axis}
                \end{tikzpicture}
                \renewcommand{\figurename}{Wykres}
                \caption{pomiary azymutu za pomocą magnetometru, oraz zmiany kąta wykazanego przez żyroskop.}
                \label{plot:gyro_magneto_measure}
        \end{figure}
        Podczas pomiaru, na oba układy został nałożony filtr dolnoprzepustowy, eliminujący szumy.

        Wykres \ref{plot:delta_angle_with_gyro}, przedstawia narastanie katą, mierzonego przez żyroskop.
        Widać bardzo dużą liniowość pracy żyroskopu, co pozwala zaufać temu czujnikowi na tyle, aby na jego podstawie określać czy pojazd jedzie prosto.

        % \begin{wrapfig}[!ht]
        \begin{figure}[!ht]
            \centering
                \begin{tikzpicture}
                    \begin{axis}[
                        width = 0.7\textwidth,
                        grid = both,
                        grid style = dashed,
                        xlabel = czas ${[s]}$,
                        ylabel = kąt zmierzony podczas obrotu ${[^\circ]}$,
                        xmin = 0,
                        xmax = 2.9,
                    ]
                        \addplot[blue] table[x = Time, y = Delta_angle, col sep = comma]{Measure/angles.csv};
                    \end{axis}
                \end{tikzpicture}
                \renewcommand{\figurename}{Wykres}
                \caption{narastania kąta, zmierzonego przez żyroskop}
                \label{plot:delta_angle_with_gyro}
        \end{figure}
        % \end{wrapfig}


    \newpage
    Oba wyżej opisane rozwiązania pozwalają na w miarę stałą jazdę prostą ($\pm 2^\circ$).
    A w sytuacjach, kiedy pojazd zaczyna znacząco skręcać, na przykład po impakcie z boku, sprzężenie zwrotne z żyroskopu pozwala wykryć znaczącą zmianę kąta ruchu.
    Kolejny regulator PID, odczytuje wartość kąta i koryguje kąt serwomechanizmu, tak aby pojazd wrócił na prostą.

    \subsection{Skręcanie}
        Rozwiązanie problemów z jazdą prosto to tylko połowa sukcesu.
        Kolejnym problemem okazało się skręcanie. 
        Nie tylko w pierwotnej wersji nie było stałe, ustawienie stałego kąta kół oraz przejechanie tej samej drogi, nie zawsze dawało ten sam rezultat.
        To pojazd cały czas miał tendencję do preferowania skrętu w jedną stronę.
        
        \subsubsection{Dyferencjał}
        \label{subsubsec:dyferencjal}
        Najprostszym do zauważenia problemem był brak dyferencjału. 
        Tylna oś pojazdu posiada dwa silniki, jednak w poprzednich testach, zostały one ,,software'owo połączona sztywną belką".
        Powodowało to poślizg pojazdu podczas skręcania a w konsekwencji uniemożliwiało powtarzalność manewru.

        Rozwiązaniem tego problemu jest zmiana prędkości pracy silników w zależności od promienia skrętu.
        Poniżej przedstawiono schemat algorytmu, na którego podstawie wyznaczono procentową wartość dyferencjału.
        Rysunek \ref{fig:turning_car}, przedstawia schematyczną sytuację skręcania.
        \begin{figure}[!ht]
    \centering
    \begin{tikzpicture}
        \draw
            (0, 0) node[draw, rectangle, rounded corners, fill = black, minimum width = 2cm, minimum height = 0.8cm, rotate = 30](frontL){}
            (0, -3) node[draw, rectangle, rounded corners, fill = black, minimum width = 2cm, minimum height = 0.8cm, rotate = 30](frontR){}

            (6, 0) node[draw, rectangle, rounded corners, fill = black, minimum width = 2cm, minimum height = 0.8cm](backL){}
            (6, -3) node[draw, rectangle, rounded corners, fill = black, minimum width = 2cm, minimum height = 0.8cm](backR){}

            (3, 0) node[above]{Zewnętrzna strona}
            (3,-3) node[below]{Wewnętrzna strona}
        ;

        \draw[Stealth-Stealth]
            (frontL) ++ (0, 1) --node[above]{$l$}++ (6, 0)
        ;
        \draw[dashed, gray]
            (frontL) --++(0, 1)
            (backL) --++(0, 1)
        ;

        \draw[color = red]
            (backL) -- ++ (0, -10) -- ++ (0, -1) coordinate(meet) --++ (0, -1)
            (frontL) -- (meet)
            (frontR) -- (meet)
        ;
        \draw
            (meet) ++ (0, 3) arc(90:127:3) node[above]{$\alpha$}
            (meet) ++ (0, 2) arc(90:147:1) node[above]{$\beta$}
        ;

        \draw[dotted]
            (backL) to[short, l=$\Delta r$] (backR)
            (backR) ++ (0, -2.5) node[right]{$r$}
        ;

        \draw[dashed]
            (frontL) ++ (0, 1) -- (0, -5)
            (0, -4) arc(90:180:-1)
            (0.5, -4.5) node[]{$\gamma$}
            (1.2, -4.25) node[rotate = -50]{$90 - \gamma$}
        ;
    \end{tikzpicture}
    \caption{Rysunek poglądowy do skręcenia}
    \label{draw:turning_car}
\end{figure}

        Jak widać na powyższym rysunku, skręt pojazdu można opisać jako ruch po kręgu, w którym poszczególne prędkości liniowe nie są sobie równe.
        Jednak występuje zależność:
        \begin{gather}
            \omega_{\text{wewnęrznego koła}} = \omega_{\text{zewnętrznego koła}}\\
            \frac{v_{\text{wewnęrznego koła}}}{r} = \frac{v_{\text{zewnętrznego koła}}}{r + \Delta r}\\
            \frac{v_\text{wewnęrznego koła}}{v_{\text{zewnętrznego koła}}} = \frac{r}{r + \Delta r}
        \end{gather}
        Znanymi wartościami są:
        \begin{itemize}
            \item $l = 155$ -- długość między osiami samochodu:,
            \item $\Delta r = 125$ -- odległość między kołami w jednej osi,
            \item $\gamma \in <60^\circ \div 120^\circ>$ -- kąt skrętu kół, ustawiany przez algorytm nawigacji lub użytkownika.
        \end{itemize}
        Dla powyższych parametrów, można wyznaczyć następujące równanie:
        \begin{gather}
            \tan \left(90 - \gamma\right) = \frac{l}{r}
        \end{gather}
        Na jego podstawie wyznaczono funkcję długości promienia skrętu od aktualnego kąta:
        \begin{gather}
            r(\gamma) = \frac{l}{\tan(90-\gamma)}
            \label{eq:turning_radius}
        \end{gather}
        A zatem procentowy stosunek prędkości w funkcji kąta można wyrazić jako:
        \begin{gather}
            \frac{v_{\text{prawe koła}}}{v_{\text{lewe koła}}} = 1 + \frac{\Delta r}{r} = 1 + \Delta r \cdot \frac{\tan(90 - \gamma)}{l}
        \end{gather}
    

    Dzięki wszystkim wyżej wymiennym zabiegom, udało się uzyskać powtarzalne wyniki zarówno podczas jazdy prosto, cofania się oraz skręcania.
    Powyższe działania są niezbędne, jeśli budowany robot ma być sterowany autonomicznie.
