\section{Podsumowanie}
    Budowa pojazdu autonomicznego, okazała się niełatwym zadaniem.
    Ilość problemów, które występują podczas prac nad takim samochodem okazała olbrzymia.
    Co gorsza napotkane problemy, nie należą do jednej dziedziny.
    Przykładowo, opisany w rozdziale \ref{subsubsec:dyferencjal} problem brakiem dyferencjały, był problemem wynikającym z mechaniki pojazdu, a rozwiązanie pochodzi przyszło od informatycznej strony.

    Z drugiej strony, najbardziej podstawowe algorytmy, będące czysto informatycznymi zagadnieniami, wymagały uwzględnienia mechaniki pojazdu.
    Przykładowo, algorytm wyznaczające trasę w pierwotnej wersji, nie zakładał dzielenia ścieżek na instrukcje a wzorował się na schemacie przedstawionym w pracy \citetitle{Simple_PathSmoothing}\cite{Simple_PathSmoothing}.
    To rozwiązanie po próbie implementacji okazało się nie wystarczające i wymusiło, poszukiwanie innych rozwiązań opisanych w rozdziale \ref{subsec:wygładzanie_ścieżek}.
    Także inne problemy, które wystąpiły podczas budowy, bardzo często pochodziły z jednej dziedziny a jedyne sensowne rozwiązanie pochodziło z drugiej.

    Podsumowując pomimo przedstawionych problemów, udało sie zbudować i odpowiednio oprogramować samochód.
    Aktualny projekt można z pełną odpowiedzialnością nazwać pojazdem antonimicznym.

