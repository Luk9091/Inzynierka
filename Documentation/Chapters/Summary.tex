\section{Podsumowanie}
    Budowa pojazdu autonomicznego okazała się niełatwym zadaniem.
    Ilość problemów, które występują podczas prac nad takim samochodem jest olbrzymia.
    Co gorsza, napotkane defekty nie należą do jednej dziedziny.
    Opisany w sekcji \ref{subsubsec:dyferencjal} brak dyferencjału wynikał z mechaniki, a~rozwiązanie zostało zaimplementowane software'owo.

    Z drugiej strony, najbardziej podstawowe algorytmy będące czysto programistycznymi zagadnieniami wymagały uwzględnienia fizyki pojazdu.
    Przykładowo, ten wyznaczający trasę w~pierwotnej wersji, nie zakładał dzielenia ścieżek na instrukcje, a wzorował się na schemacie zaprezentowanym w artykule \citetitle{Simple_PathSmoothing}\cite{Simple_PathSmoothing}.
    Po próbie uruchomienia ta idea była niewystarczająca i wymusiła poszukiwanie innych rozwiązań opisanych w rozdziale \ref{subsec:wygładzanie_ścieżek}.
    Także inne sytuacje, które wystąpiły podczas budowy, bardzo często wywodziły się z jednego działu, a jedyna sensowna odpowiedź pochodziła z drugiego.

    Podsumowując, pomimo przedstawionych trudności udało się zbudować i odpowiednio oprogramować samochód.
    Aktualny projekt można z pełną odpowiedzialnością nazwać pojazdem autonomicznym.

