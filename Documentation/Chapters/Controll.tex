\section{Sterowanie pojazdem}
    Sterowanie pojazdem powinno odbywać się w sposób płynny i precyzyjny.
    W poniższym rozdziale zostaną omówione algorytmy odpowiedzialne za najbardziej podstawowe funkcje poruszania się takie jak: jazda prosto, skręt oraz cofanie.

    \subsection{Jazda przez pewien odcinek}
    \label{subsec:jazda_przez_odcinek}
        W rozdziale \ref{section:jazda_prosto} zostały opisane problemy jakie wystąpiły podczas budowy pojazdu. %oraz problemy jakie musiały zostać rozwiązane aby samochód był w stanie jechać prosto.
        Dzięki ich rozwiązaniu możliwa była jazda prosto.
        Jednak dla precyzyjnego sterowania informacja o tym, że pojazd porusza się po linii prostej jest niewystarczająca.
        Obowiązkowym staje się poznanie odległości jaka została pokonana.

        Dystans jaki pokonuje pojazd od momentu startu, możemy dość dokładnie obliczyć korzystając ze wzoru \eqref{eq:pulseToDistance}.
        \begin{gather}
            s = \frac{\text{pulse} \cdot 2\pi R}{N}
            \label{eq:pulseToDistance}
        \end{gather}
        gdzie:
        \begin{itemize}
            \item $s$ -- odległość jaką pokonał pojazd,
            \item $\text{pulse}$ -- liczba impulsów z enkodera,
            \item $R$ -- promień koła,
            \item $N$ -- liczba impulsów na obrót koła.
        \end{itemize}

        Wzór ten można także przekształcić w drugą stronę, aby obliczyć ile impulsów enkodera musi zostać zliczonych przez procesor aby pojazd pokonał zadaną odległość.
        \begin{gather}
            \text{pulse} = \frac{s \cdot N}{2\pi R}
            \label{eq:distanceToPulse}
        \end{gather}

        Dla zbudowanego modelu promień koła wynosi $R = 50cm$ a liczba impulsów zgodnie z dokumentacją wynosi $N = 1920$.
        A więc dokładność pomiaru odległości wynosi około:
        \begin{gather}
            \Delta s \approx \pm2.0mm
        \end{gather}

        Dzięki zastosowaniu równania \eqref{eq:distanceToPulse}, możliwe jest wysłanie informacji do pojazdu aby po określonej ilości impulsów z enkoderów, zatrzymał się.


    \subsection{Wyznaczanie zakrętów}
        Wyznaczenie idealnie prostej trasy dla pojazdów nie zawsze jest możliwe.
        W trakcie jazdy, samochód będzie skręcał wielotonie, w każdym możliwym kierunku.
        Przedstawiony model posiada dwie osiowe, przez co nie jest w stanie wykonać tego manewru w miejscu.
        Natomiast może swobodnie poruszać się po okręgu o minimalnym promieniu.
        W tym rozdziale zostanie opisany sposób wyznaczania tego promienia w zależności od kąta skrętu kół.

        \subsubsection{Minimalny promień skrętu}
        \label{subsubsec:minamalny_promien}
            Wykorzystując rysunek \ref{draw:turning_car} oraz zależność \eqref{eq:turning_radius}, można obliczyć minimalny promień skrętu.
            Po podstawieniu wartości do wzoru, otrzymujemy:
            \begin{gather}
                r(\gamma = 90 \pm 30^\circ) = \left|\frac{155}{\tan(\pm 30^\circ)}\right| \approx 270mm
                \label{eq:theoretical_radius}
            \end{gather}

            Otrzymaną wartość należy jednak skonfrontować z rzeczywistością.
            W tym celu przedstawiono inny sposób wyznaczenia promienia skrętu.
            Równanie \eqref{eq:turning_arc} pozwala na obliczenie tego parametru na podstawie zakreślonego kąta i długości łuku.
            \begin{gather}
                s = 2\pi (r(\gamma) + \Delta r) \cdot \frac{\alpha}{360}
                \label{eq:turning_arc}
                % \\
                % \alpha = \frac{2\pi (r(\gamma + \Delta r))}{s \cdot 360}
            \end{gather}
            gdzie:
            \begin{itemize}
                \item $s$ -- długość łuku,
                \item $r(\gamma)$ -- promień skrętu dla danego skrętu kół,
                \item $\Delta r$ -- odległość między kołami,
                \item $\alpha$ -- oczekiwany kąt skrętu.
            \end{itemize}

            Przekształcając powyższe równanie otrzymujemy zależność:
            \begin{gather}
                r(\gamma) = \frac{s}{2\pi \cdot \frac{\alpha}{360}} - \Delta r
                \label{eq:turning_radius_with_arc}
            \end{gather}

            Odległość pokonana przez samochód jest dość precyzyjne mierzona przez enkodery.
            A zakreślony kąt został zmierzony przez wcześniej skalibrowany akcelerometr.
            Dzięki czemu można uznać, że zadana długość łuku i zakreślony kat są precyzyjne.

            Przykładowo dla ustawionego maksymalnego kąta skrętu kół $\gamma = 90 + 30^\circ$ oraz długości łuku $s = 500mm$, wartość zakreślonego kąta wynosiła około $\alpha = 60.0^\circ$.
            Podstawiając podane wyniki do równania \eqref{eq:turning_radius_with_arc}, otrzymujemy:
            \begin{gather}
                r = \frac{500mm}{2\pi \cdot \frac{60}{360}} - 125mm \approx (350)mm
            \end{gather}


            Teoretyczny minimalny promień skrętu wychodzący z obliczeń (równanie \eqref{eq:theoretical_radius}), jest znacząco mniejszy od zmierzonego minimalnego promienia skrętu.
            Wynika to z nie idealności konstrukcji oraz zastosowania ,,względnie słabej jakości'' serwomechanizmu.
            Co ogranicza maksymalny zakres skrętu do $\pm 24^\circ$.
            \begin{gather}
                r(\gamma = 90 + 24^\circ) = \left|\frac{155}{\tan(+ 24^\circ)}\right| \approx 350mm
            \end{gather}

            Jednak po uwzględnieniu zależności z rozdziału \ref{subsec:nierownomiernosc_skretu}, okazuje się że różnica między teoretycznym a obliczonym kątem skrętu kół wynosi dokładnie wartość offsetu ($-6^\circ$).
            Dlatego też ograniczenie skrętu kół do $\pm 24^\circ$ nie jest potrzebne. 
            I w późniejszych obliczeniach kąt skrętu kół brany pod uwagę będzie w zakresie $\pm 30^\circ$.


