\section{Sterowanie pojazdem}
    Kierowanie pojazdem powinno odbywać się w sposób płynny i precyzyjny.
    W niniejszym rozdziale zostaną omówione algorytmy, odpowiedzialne za najbardziej podstawowe funkcje poruszania się, takie jak: jazda prosto, skręt oraz cofanie.

    \subsection{Jazda przez pewien odcinek}
    \label{subsec:jazda_przez_odcinek}
        Poprzednio (rozdz. \ref{sec:problemy_konstrukcyjne}) opisano problemy, jakie wystąpiły podczas budowy projektu.
        Ich rozwiązanie umożliwiło powtarzalność manewrów.
        Jednak dla precyzyjnego sterowania informacja o~tym, że pojazd porusza się po linii jest niewystarczająca.
        Obowiązkowym staje się poznanie odległości jaka została pokonana.

        Dystans przejechany przez samochód od momentu startu, możemy dość dokładnie obliczyć korzystając ze wzoru \eqref{eq:pulseToDistance}.
        \begin{gather}
            s = \frac{pulse \cdot 2\pi R}{N}
            \label{eq:pulseToDistance}
        \end{gather}
        gdzie:
        \begin{itemize}
            \item $s$ -- odległość pokonana,
            \item $pulse$ -- liczba impulsów z enkodera,
            \item $R$ -- promień koła,
            \item $N$ -- liczba impulsów na obrót.
        \end{itemize}

        Przekształcenie tego równania, pozwala na wyznaczenie ilości impulsów, które muszą być zliczone na danym dystansie.
        % Równanie to można także przekształcić w drugą stronę, aby wyznaczyć ile impulsów enkodera musi zostać zliczonych przez procesor, aby pojazd pokonał zadaną odległość.
        \begin{gather}
            \text{pulse} = \frac{s \cdot N}{2\pi R}
            \label{eq:distanceToPulse}
        \end{gather}

        Dla zbudowanego modelu promień koła jest równy $R = 25 mm$, a liczba impulsów zgodnie z dokumentacją to $N = 1920$.
        A więc dokładność pomiaru odległości wynosi około:
        \begin{gather}
            \Delta s \approx \pm0.1mm
        \end{gather}

        Wykorzystując formułę \eqref{eq:distanceToPulse}, procesor jest w stanie zatrzymać silniki, po przejechaniu określonej drogi.


    \subsection{Wyznaczanie zakrętów}
    \label{subsubsec:minamalny_promien}
        Opracowanie idealnie prostej trasy dla pojazdów nie zawsze jest możliwe.
        W trakcie jazdy samochód będzie skręcał wielokrotnie w każdym kierunku.
        Przedstawiony model posiada dwie osie, przez co nieosiągalne jest wykonanie tego manewru w miejscu.
        Natomiast może swobodnie poruszać się po okręgu o minimalnym promieniu.
        W tej sekcji zostanie opisany sposób jego wyznaczania w funkcji od kąta skrętu kół.

\newpage
        % \subsubsection{Minimalny promień skrętu}
            Wykorzystując rysunek \ref{draw:turning_car} oraz zależność \eqref{eq:turning_radius}, można obliczyć promień skrętu.
            Po podstawieniu wartości do wzoru, otrzymujemy:
            \begin{gather}
                r(\gamma = 90 \pm 30^\circ) = \left|\frac{155}{\tan(\pm 30^\circ)}\right| \approx 270mm
                \label{eq:theoretical_radius}
            \end{gather}

            Obliczoną wartość należy skonfrontować z rzeczywistością.
            W tym celu zaprezentowano inną metodę wyznaczenia tego samego promienia.
            Równanie \eqref{eq:turning_arc} pozwala na poznanie omawianego parametru na podstawie zakreślonego kąta i długości łuku.
            \begin{gather}
                s = 2\pi (r(\gamma) + \Delta r) \cdot \frac{\alpha}{360}
                \label{eq:turning_arc}
                % \\
                % \alpha = \frac{2\pi (r(\gamma + \Delta r))}{s \cdot 360}
            \end{gather}
            gdzie:
            \begin{itemize}
                \item $s$ -- długość łuku,
                \item $r(\gamma)$ -- promień skrętu,
                \item $\Delta r$ -- odległość między kołami,
                \item $\alpha$ -- zakreślony kąt.
            \end{itemize}

            Przekształcając powyższą zależność uzyskujemy:
            \begin{gather}
                r(\gamma) = \frac{s}{2\pi \cdot \frac{\alpha}{360}} - \Delta r
                \label{eq:turning_radius_with_arc}
            \end{gather}

            Odległość pokonana przez samochód jest dokładnie mierzona przez enkodery, a zakreślony kąt został wyznaczony przez wcześniej skalibrowany akcelerometr.
            Z tego względu można uznać, że obie wartości są precyzyjne.

            Przykładowo, dla ustawionego maksymalnego kąta skrętu kół $\gamma = (90 + 30)^\circ$ oraz długości łuku $s = 500mm$, zakreślony kąt wyniósł $\alpha = 60.0^\circ$.
            Podstawiając podane wyniki do równania \eqref{eq:turning_radius_with_arc}, otrzymujemy:
            \begin{gather}
                r = \frac{500mm}{2\pi \cdot \frac{60}{360}} - 125mm \approx 350mm
            \end{gather}


            Teoretyczny minimalny promień skrętu wychodzący z obliczeń \eqref{eq:theoretical_radius}, jest znacząco mniejszy od zmierzonego.
            Wynika to z nieidealności konstrukcji oraz zastosowania ,,względnie słabej jakości'' serwomechanizmu,
            co ogranicza maksymalny zakres skrętu do $\pm 24^\circ$.
            \begin{gather}
                r(\gamma = 90 + 24^\circ) = \left|\frac{155}{\tan(+ 24^\circ)}\right| \approx 350mm
            \end{gather}

            Jednak po uwzględnieniu zależności z sekcji \ref{subsec:nierownomiernosc_skretu} okazuje się, że różnica między teoretycznym a obliczonym kątem skrętu kół wynosi dokładnie wartość offsetu ($-6^\circ$).
            Dlatego też ograniczenie skrętu kół do $\pm 24^\circ$ nie jest potrzebne.
            W późniejszych obliczeniach kąt skrętu kół brany pod uwagę będzie w zakresie $\pm 30^\circ$.


