    \useNormalLandscape
\lsection{Schemat blokowy}
    Na schemacie blokowym \ref{schema:block}, przedstawiono przepływ sygnałów sterujących, wraz z poszczególnymi blokami.
\begin{figure}[!ht]
\centering
\vspace{1cm}
\begin{circuitikz}[fill = white]
    \draw[very thick]
        (0, 0) node[draw, rectangle, fill, minimum width = 4cm, minimum height = 4cm, align = center] (MCU) {Mikrokontroler\\\scriptsize{Raspberry PI PICO W}}

        (0, -5) node[draw, rectangle, fill, minimum width = 3cm, minimum height = 3cm, align = center] (HBridge){Mostek H}
        (HBridge.south) ++(1, 0) --++(0, -1) to[Telmech=M, n = LMotor] ++ (2, 0) --++ (0, 2) --++ (-2, 0)
        (HBridge.south) ++(-1, 0) --++(0, -1) to[Telmech=M,n = RMotor] ++ (-2, 0) --++ (0, 2) --++ (2, 0)

        (-10, 4) node[draw, rectangle, fill, minimum width = 2cm, minimum height = 2cm, align = center] (ToF1) {$\text{ToF}_1$}
        (-7, 4) node[draw, rectangle, fill, minimum width = 2cm, minimum height = 2cm, align = center] (ToF2) {$\text{ToF}_2$}
        (-4, 4) node[draw, rectangle, fill, minimum width = 2cm, minimum height = 2cm, align = center] (ToF3) {$\text{ToF}_3$}

        (-6, 0) node[draw, rectangle, fill, minimum width = 2cm, minimum height = 2cm, align = center] (MPU6050){MPU6050\\lub\\MPU6000}
        (-6, -4) node[draw, rectangle, fill, minimum width = 2cm, minimum height = 2cm, align = center] (Magneto){Magnetometr}

        (6, -4) node[draw, rectangle, fill, minimum width = 2cm, minimum height = 2cm, align = center] (LIR){Czujnik\\cofania}
        (9, -4) node[draw, rectangle, fill, minimum width = 2cm, minimum height = 2cm, align = center] (RIR){Czujnik\\cofania}

        (6, 4) node[elmech=M, scale = 1.5](servo){S}
        (8, 0) node[draw, rectangle, fill, minimum width = 2cm, minimum height = 2cm, align = center] (memory) {Pamięć}
    ;

    \draw[]
        (MCU.east) to[bmultiwire = 4, a = SPI] (memory.west)
        (MCU.east) ++ (0, 1) to[short, i=PWM] ++ (4, 0) -| (servo.south)
        (MCU.east) ++ (0,-1) to[bmultiwire = 2] ++ (2, 0) --++ (0, -1) to[short, i<= Detect, -*]++ (2, 0) coordinate(IR_node)
            (IR_node)  to[multiwire=1] (LIR.north)
            (IR_node)  to[multiwire=1] ++ (3, 0)-| (RIR.north)
    
        (MCU.south) to[tmultiwire=6, i=\ ] (HBridge.north)

        (MCU.west) to[bmultiwire = 2, a = I2C, -*] ++ (-2, 0) coordinate(I2C) 
        (I2C) -- (MPU6050.east)
        (I2C) |- (Magneto)

        (I2C) to[short, -*] ++ (0, 2) coordinate(I2C)
        (I2C) -- (ToF3.south)
        (I2C) to[short, -*]++ (-3, 0) coordinate(I2C) -- (ToF2.south)
        (I2C) -| (ToF1)

        (LMotor) to[crossing] ++ (0, 4) --++ (0, 0.5) to[bmultiwire, a = 2] ++ (-1, 0) --++ (0, 1)
        (RMotor) to[crossing] ++ (0, 4) --++ (0, 0.5) to[bmultiwire = 2] ++ ( 1, 0) --++ (0, 1)
    ;
\end{circuitikz}
\renewcommand{\figurename}{Schemat}
\caption{Schemat blokowy, przedstawiający przepływ sygnałów w projekcie}
\label{schema:block}
\end{figure}
    \useportrait
% \footnote{Pamięć dodatkowa do przechowywania map terenu}
